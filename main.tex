\documentclass{book}
\usepackage[brazil,latin]{babel}
\usepackage[T1]{fontenc}
\usepackage{geometry}
\usepackage{color}
\usepackage[autocompile]{gregoriotex}
\usepackage{lettrine}

\geometry{a4paper,hdivide={1.5cm,*,1.5cm}, vdivide={1.5cm,*,1.5cm}}

\begin{document}

\pagestyle{empty}

\begin{center}

    \huge Arquidiocese de Olinda e Recife
    \vspace{0.3cm} \\
    \LARGE Vicariato Paulista
    \vspace{3cm} \\
    \textcolor{red}{\Huge Domingo de Pentecostes}
    \vspace{3cm} \\
    \huge Missa da Vigília
    \vspace{0.3cm} \\
    \LARGE Presidida por Monsenhor
    \vspace{0.3cm} \\
    \textcolor{red}{\Huge Oswaldo Lopes}
    \vspace{\fill}\\
    \LARGE Paulista, 28 de Maio de 2023

\end{center}

\newpage

\begin{center}

    \textbf{Ritos Iniciais}

\end{center}

\begin{flushleft}

    \vspace{0.5cm}
    \textcolor{red}{Antífona de Entrada}
    \hspace{\fill}
    \textcolor{red}{Rm 5,5; 10,11}
    \vspace{0.2cm} \\
    O amor de Deus foi derramado em nossos corações \\
    pelo seu Espírito que habita em nós, aleluia!
    \vspace{0.2cm} \\
    \textcolor{red}{Saudação}
    \vspace{0.2cm}\\
    Em nome do Pai e do Filho e do Espírito Santo. \\
    {\color{red} \Rbar.} Amém. \\
    A graça de nosso Senhor Jesus Cristo, \\
    amor do pai \\
    e a comunhão do Espírito Santo \\
    estejam convosco. \\
    {\color{red} \Rbar.} Bendito seja Deus \\
    que nos reuniu no amor de Cristo.
    \vspace{0.2cm} \\
    \textcolor{red}{Ato Penitencial}
    \vspace{0.2cm} \\
    \textcolor{red}{Confesso}
    \vspace{0.2cm} \\
    No dia em que celebramos a vitória de Cristo \\
    sobre o pecado e a morte, \\
    também nós somos convidados a morrer para o pecado \\
    e ressurgir para uma vida nova. \\
    Reconheçamo-nos necessitados da misericórdia do Pai.
    \vspace{0.1cm} \\
    Confessemos os nossos pecados:
    \vspace{0.1cm} \\
    Confesso a Deus todo-poderoso \\
    e a vós, irmãos e irmãs, \\
    que pequei muitas vezes \\
    por pensamentos e palavras, \\
    atos e omissões, \\
    por minha culpa, minha tão grande culpa \\
    E peço à Virgem Maria, \\
    aos anjos e santos \\
    e a vós, irmãos e irmãs, \\
    que roqueis por mim a Deus, nosso Senhor.
    \vspace{0.1cm} \\
    Deus todo-poderoso tenha compaixão de nós, \\
    perdoe os nossos pecados \\
    e nos conduz à vida eterna. \\
    {\color{red} \Rbar.} Amém.
    \vspace{0.2cm} \\
    \textcolor{red}{Kyrie}
    \vspace{0.2cm} \\
    {\color{red} \Vbar.} Senhor, tende piedade de nós \\
    {\color{red} \Rbar.} Senhor, tende piedade de nós
    \vspace{0.1cm} \\
    {\color{red} \Vbar.} Cristo, tende piedade de nós \\
    {\color{red} \Rbar.} Cristo, tende piedade de nós
    \vspace{0.1cm} \\
    {\color{red} \Vbar.} Senhor, tende piedade de nós \\
    {\color{red} \Rbar.} Senhor, tende piedade de nós
    \vspace{0.2cm} \\
    \textcolor{red}{Oração}
    \vspace{0.1cm} \\
    Oremos.
    \vspace{0.1cm}\\
    Concedei-nos, ó Deus onipotente, \\
    que brilhe sobre nós o esplendor da vossa claridade, \\
    e o fulgor da vossa luz confirme, \\
    com o dom do Espírito Santo, \\
    aqueles que renasceram pela vossa graça. \\
    Por nossa Senhor Jesus Cristo, vosso Filho, \\
    na unidade do Espírito Santo. \\
    {\color{red} \Rbar.} Amém. \\

    \newpage

    \textcolor{red}{Exortação}
    \vspace{0.2cm} \\
    Introduzidos na Vigília de Pentecostes, irmãs e irmãos caríssimos, a exemplo dos Apóstolos e discípulos que, com Maria, a Mãe de jesus, perseveravam em oração, aguardando o Espírito prometido pelo Senhor, ouçamos, de ânimo sereno, a Palavra de Deus. \\
    Meditemos sobre as grandes coisas que Deus realizou em favor de seu povo e rezemos, para que o Espirito Santo, enviado pelo Pai como primícias aos que creem, leve à plenitude a sua abra neste mundo.

\end{flushleft}

\begin{center}

    \textbf{Liturgia da Palavra}
    \vspace{0.5cm}\\
    \textcolor{red}{Primeira Leitura}

\end{center}

\begin{flushright}
    \textit{Foi chamada Babel, porque foi aí que o Senhor \\ confundiu a linguagem de todo o mundo.}
\end{flushright}

\begin{flushleft}

    \vspace{0.2cm}
    Leitura do Livro do Gênesis
    \hspace{\fill}
    \textcolor{red}{11,1--9}
    \vspace{0.2cm} \\
    Toda a terra tinha um só linguagem \\
    e servia-se das mesmas palavras. \\
    E aconteceu que, partindo do oriente, \\
    os homens acharam uma planície na terra de Senaar, \\
    e aí se estabeleceram. \\
    E disseram uns aos outros: \\
    ``Vamos, façamos tijolos e cozamos-los ao fogo''. \\
    Usaram tijolos em vez de pedra, \\
    e betume em lugar de argamassa. \\
    e Disseram: \\
    ``Vamos, façamos para nós uam cidade \\
    e uma torre cujo cimo atinja o céu. \\
    Assim, ficaremos famosos, \\
    e não seremos dispersos por toda a face da terra''. \\
    Então o Senhor desceu para ver a cidade \\
    e a torre que os homens estavam construindo. \\
    E o Senhor disse: \\
    ``Eis que eles são um só povo e falam uma só língua, \\
    Agora, nada os impedirá de fazer o que se propuseram. \\
    Desçamos e confundamos a sua língua \\
    de modo que não se entendam uns aos outros''. \\
    E o Senhor os dispersou daquele lugar \\
    por toda a superfície da terra, \\
    e eles cessaram de construir a cidade. \\
    Por isso, foi chamada Babel, porque foi aí que o Senhor \\
    confundiu a linguagem de todo o mundo, \\
    e daí dispersou os homens por toda terra.
    \vspace{0.1cm} \\
    Palavra do Senhor \\
    {\color{red} \Rbar.} Graças a Deus.
    \vspace{0.2cm} \\
    \textcolor{red}{Salmo Responsorial
        \hspace{\fill} Sl 32(33),10--11.12--13.14--15 \\
        \hspace{\fill}
        ({\color{red} \Rbar.} 12b)} \\
    \vspace{0.1cm}
    {\color{red} \Rbar.} Feliz o povo que o Senhor escolheu por sua herança!
    \vspace{0.1cm} \\
    1. O Senhor desfaz os planos das nações \\
    e os projetos que os povos se propõem. \\
    Mas os desígnios do Senhor são para sempre, \\
    e os pensamentos que ele traz no coração, \\
    de geração em geração, vão perdurar.
    \hspace{\fill}{\color{red} \Rbar.}
    \vspace{0.1cm} \\
    2. Feliz o povo cujo Deus é o Senhor, \\
    e a nação que escolheu por sua herança! \\
    Dos altos céus o Senhor olha e observa; \\
    ele se inclina para olhar todos os homens.
    \hspace{\fill}{\color{red} \Rbar.}
    \vspace{0.1cm} \\
    3. Ele contempla do lugar onde reside \\
    e vê a todos os que habitam sobre a terra. \\
    Ele formou o coração de cada um \\
    e por todos os seus atos se interessa.
    \hspace{\fill}{\color{red} \Rbar.} \\

    \newpage

    \textcolor{red}{Oração}
    \vspace{0.2cm} \\
    \textcolor{red}{Glória}
    \vspace{0.2cm} \\
    Glória a Deus nas alturas, \\
    e paz na terra aos homens por Ele amados. \\
    Senhor Deus, rei dos céus, \\
    Deus Pai todo-poderoso: \\
    nós vos louvamos, \\
    nós vos adoramos, \\
    nós vos glorificamos, \\
    nós vos damos graça \\
    por vossa imensa glória. \\
    Senhor Jesus Cristo, Filho Unigênito, \\
    Senhor Deus, Cordeiro de Deus, \\
    Filho de Deus Pai. \\
    Vós que tirais o pecado do mundo, \\
    tende piedade de nós. \\
    Vós que tirais o pecado do mundo, \\
    acolhei a nossa súplica. \\
    Vós que estais à direita do Pai, \\
    tende piedade de nós. \\
    Só vós sois o Santo, \\
    só vós, o Senhor, \\
    só vós, o Altissimo, \\
    Jesus Cristo, \\
    com o Espírito Santo, \\
    na glória de Deus Pai. \\
    Amém.

\end{flushleft}

\end{document}

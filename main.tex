\documentclass{book}
\input{packages}
\geometry{a4paper,hdivide={1.5cm,*,1.5cm}, vdivide={1.5cm,*,1.5cm}}

\begin{document}

\pagestyle{empty}

\begin{center}

    \huge Arquidiocese de Olinda e Recife
    \vspace{0.3cm} \\
    \LARGE Vicariato Paulista
    \vspace{3cm} \\
    \textcolor{red}{\Huge Domingo de Pentecostes}
    \vspace{3cm} \\
    \huge Missa da Vigília
    \vspace{0.3cm} \\
    \LARGE Presidida por Monsenhor
    \vspace{0.3cm} \\
    \textcolor{red}{\Huge Oswaldo Lopes}
    \vspace{\fill}\\
    \LARGE Paulista, 28 de Maio de 2023

\end{center}

\newpage

\begin{center}

    \textbf{Ritos Iniciais}

\end{center}

\begin{flushleft}

    \vspace{0.5cm}
    \textcolor{red}{Antífona de Entrada}
    \hspace{\fill}
    \textcolor{red}{Rm 5,5; 10,11}
    \vspace{0.2cm} \\
    O amor de Deus foi derramado em nossos corações \\
    pelo seu Espírito que habita em nós, aleluia!
    \vspace{0.2cm} \\
    \textcolor{red}{Saudação}
    \vspace{0.2cm}\\
    Em nome do Pai e do Filho e do Espírito Santo. \\
    {\color{red} \Rbar.} Amém. \\
    A graça de nosso Senhor Jesus Cristo, \\
    amor do pai \\
    e a comunhão do Espírito Santo \\
    estejam convosco. \\
    {\color{red} \Rbar.} Bendito seja Deus \\
    que nos reuniu no amor de Cristo.
    \vspace{0.2cm} \\
    \textcolor{red}{Ato Penitencial}
    \vspace{0.2cm} \\
    \textcolor{red}{Confesso}
    \vspace{0.2cm} \\
    No dia em que celebramos a vitória de Cristo \\
    sobre o pecado e a morte, \\
    também nós somos convidados a morrer para o pecado \\
    e ressurgir para uma vida nova. \\
    Reconheçamo-nos necessitados da misericórdia do Pai.
    \vspace{0.1cm} \\
    Confessemos os nossos pecados:
    \vspace{0.1cm} \\
    Confesso a Deus todo-poderoso \\
    e a vós, irmãos e irmãs, \\
    que pequei muitas vezes \\
    por pensamentos e palavras, \\
    atos e omissões, \\
    por minha culpa, minha tão grande culpa \\
    E peço à Virgem Maria, \\
    aos anjos e santos \\
    e a vós, irmãos e irmãs, \\
    que roqueis por mim a Deus, nosso Senhor.
    \vspace{0.1cm} \\
    Deus todo-poderoso tenha compaixão de nós, \\
    perdoe os nossos pecados \\
    e nos conduz à vida eterna. \\
    {\color{red} \Rbar.} Amém.
    \vspace{0.2cm} \\
    \textcolor{red}{Kyrie}
    \vspace{0.2cm} \\
    {\color{red} \Vbar.} Senhor, tende piedade de nós \\
    {\color{red} \Rbar.} Senhor, tende piedade de nós
    \vspace{0.1cm} \\
    {\color{red} \Vbar.} Cristo, tende piedade de nós \\
    {\color{red} \Rbar.} Cristo, tende piedade de nós
    \vspace{0.1cm} \\
    {\color{red} \Vbar.} Senhor, tende piedade de nós \\
    {\color{red} \Rbar.} Senhor, tende piedade de nós
    \vspace{0.2cm} \\
    \textcolor{red}{Oração}
    \vspace{0.1cm} \\
    Oremos.
    \vspace{0.1cm}\\
    Concedei-nos, ó Deus onipotente, \\
    que brilhe sobre nós o esplendor da vossa claridade, \\
    e o fulgor da vossa luz confirme, \\
    com o dom do Espírito Santo, \\
    aqueles que renasceram pela vossa graça. \\
    Por nossa Senhor Jesus Cristo, vosso Filho, \\
    na unidade do Espírito Santo. \\
    {\color{red} \Rbar.} Amém. \\

    \newpage

    \textcolor{red}{Exortação}
    \vspace{0.2cm} \\
    Introduzidos na Vigília de Pentecostes, irmãs e irmãos caríssimos, a exemplo dos Apóstolos e discípulos que, com Maria, a Mãe de jesus, perseveravam em oração, aguardando o Espírito prometido pelo Senhor, ouçamos, de ânimo sereno, a Palavra de Deus. \\
    Meditemos sobre as grandes coisas que Deus realizou em favor de seu povo e rezemos, para que o Espirito Santo, enviado pelo Pai como primícias aos que creem, leve à plenitude a sua abra neste mundo.

\end{flushleft}

\begin{center}

    \textbf{Liturgia da Palavra}
    \vspace{0.5cm}\\
    \textcolor{red}{Primeira Leitura}

\end{center}

\begin{flushright}
    \textit{Foi chamada Babel, porque foi aí que o Senhor \\ confundiu a linguagem de todo o mundo.}
\end{flushright}

\begin{flushleft}

    \vspace{0.2cm}
    Leitura do Livro do Gênesis
    \hspace{\fill}
    \textcolor{red}{11,1--9}
    \vspace{0.2cm} \\
    Toda a terra tinha um só linguagem \\
    e servia-se das mesmas palavras. \\
    E aconteceu que, partindo do oriente, \\
    os homens acharam uma planície na terra de Senaar, \\
    e aí se estabeleceram. \\
    E disseram uns aos outros: \\
    ``Vamos, façamos tijolos e cozamos-los ao fogo''. \\
    Usaram tijolos em vez de pedra, \\
    e betume em lugar de argamassa. \\
    e Disseram: \\
    ``Vamos, façamos para nós uam cidade \\
    e uma torre cujo cimo atinja o céu. \\
    Assim, ficaremos famosos, \\
    e não seremos dispersos por toda a face da terra''. \\
    Então o Senhor desceu para ver a cidade \\
    e a torre que os homens estavam construindo. \\
    E o Senhor disse: \\
    ``Eis que eles são um só povo e falam uma só língua, \\
    Agora, nada os impedirá de fazer o que se propuseram. \\
    Desçamos e confundamos a sua língua \\
    de modo que não se entendam uns aos outros''. \\
    E o Senhor os dispersou daquele lugar \\
    por toda a superfície da terra, \\
    e eles cessaram de construir a cidade. \\
    Por isso, foi chamada Babel, porque foi aí que o Senhor \\
    confundiu a linguagem de todo o mundo, \\
    e daí dispersou os homens por toda terra.
    \vspace{0.1cm} \\
    Palavra do Senhor \\
    {\color{red} \Rbar.} Graças a Deus.
    \vspace{0.2cm} \\
    \textcolor{red}{Salmo Responsorial
        \hspace{\fill} Sl 32(33),10--11.12--13.14--15}
    \vspace{0.1cm} \\
    {\color{red} \Rbar.} Feliz o povo que o Senhor escolheu por sua herança!
    \hspace{\fill}
    {\color{red} (\Rbar. 12b)}
    \vspace{0.1cm} \\
    1. O Senhor desfaz os planos das nações \\
    e os projetos que os povos se propõem. \\
    Mas os desígnios do Senhor são para sempre, \\
    e os pensamentos que ele traz no coração, \\
    de geração em geração, vão perdurar.
    \hspace{\fill}{\color{red} \Rbar.}
    \vspace{0.1cm} \\
    2. Feliz o povo cujo Deus é o Senhor, \\
    e a nação que escolheu por sua herança! \\
    Dos altos céus o Senhor olha e observa; \\
    ele se inclina para olhar todos os homens.
    \hspace{\fill}{\color{red} \Rbar.}
    \vspace{0.1cm} \\
    3. Ele contempla do lugar onde reside \\
    e vê a todos os que habitam sobre a terra. \\
    Ele formou o coração de cada um \\
    e por todos os seus atos se interessa.
    \hspace{\fill}{\color{red} \Rbar.}

    \newpage

    \textcolor{red}{Oração}
    \vspace{0.1cm} \\
    Oremos.
    \vspace{0.1cm}\\
    Ó Deus todo-poderoso, \\
    que a vossa Igreja seja sempre aquele povo santo, \\
    reunido na unidade do Pai, do Filho e do Espírito Santo, \\
    que manifesta ao mundo \\
    o mistério da vossa santidade e unidade, \\
    levando as pessoas à perfeição do vosso amor. \\
    Por Cristo, nosso Senhor. \\
    {\color{red} \Rbar.} Amém.
    \vspace{0.2cm} \\

\end{flushleft}

\begin{center}

    \textcolor{red}{Segunda Leitura}

\end{center}

\begin{flushright}
    \textit{O Senhor desceu sobre o monte Sinai \\ diante de todo o povo.}
\end{flushright}

\begin{flushleft}

    \vspace{0.2cm}
    Leitura do Livro do Êxodo
    \hspace{\fill}
    \textcolor{red}{19,3--8a.16--20b}
    \vspace{0.2cm} \\
    Naqueles dias, \\
    Moisés subiu ao encontro de Deus. \\
    O Senhor chamou-o do alto da montanha, e disse: \\
    ``Assim deverás falar à casa de Jacó \\
    e anunciar aos filhos de Israel: \\
    Vistes o que fiz aos egípcios, \\
    e como vos levei sobre asas de águia \\
    e vos trouxe a mim. \\
    Portanto, se ouvirdes a minha voz \\
    e guardardes a minha aliança, \\
    sereis para mim a porção escolhida \\
    dentre todos os povos, porque minha é toda a terra. \\
    E vós sereis para mim um reino de sacerdotes \\
    e uma nação santa. \\
    São estas as palavras \\
    que deverás dizer aos filhos de Israel''. \\
    Moisés voltou e, convocando os anciãos do povo, \\
    expôs tudo o que o Senhor lhe tinha mandado. \\
    E o povo todo respondeu a uma só voz: \\
    ``Faremos tudo o que o Senhor disse''. \\
    Quando chegou o terceiro dia, ao raiar da manhã, \\
    houve trovões e relâmpagos. \\
    Uma nuvem espessa cobriu a montanha, \\
    e um fortíssimo som de trombetas e se fez ouvir. \\
    No acampamento o povo se pôs a tremer. \\
    Moisés fez o povo sair do acampamento ao encontro de Deus, \\
    e eles pararam ao pé da montanha. \\
    Todo o monte Sinai fumegava, \\
    pois o Senhor descera sobre ele me meio ao fogo. \\
    A fumaça subia como de uma fornalha, \\
    e todo o monte tremia violentamente. \\
    O som da trombeta ia aumentando cada vez mais. \\
    Moisés falava \\
    e o Senhor lhe respondia através do trovão. \\
    O Senhor desceu sobre o monte Sinai \\
    e chamou Moisés ao cume do monte.
    \vspace{0.1cm} \\
    Palavra do Senhor \\
    {\color{red} \Rbar.} Graças a Deus.

    \newpage

    \textcolor{red}{Salmo Responsorial
        \hspace{\fill} Dn 3,52.53.54--55.56b}
    \vspace{0.1cm} \\
    {\color{red} \Rbar.} A vós louvor, honra e glória eternamente!
    \hspace{\fill}
    {\color{red} (\Rbar. 52b)}
    \vspace{0.1cm} \\
    1. Sede bendito, Senhor Deus de nossos pais.
    \hspace{\fill}{\color{red} \Rbar.}
    \vspace{0.1cm} \\
    2. Sede bendito, nome santo e glorioso.
    \hspace{\fill}{\color{red} \Rbar.}
    \vspace{0.1cm} \\
    3. No templo santo, onde refulge a vossa glória.
    \hspace{\fill}{\color{red} \Rbar.}
    \vspace{0.1cm} \\
    4. E em vosso trono de poder vitorioso.
    \hspace{\fill}{\color{red} \Rbar.}
    \vspace{0.1cm} \\
    5. Sede bendito, que sondais as profundezas.
    \hspace{\fill}{\color{red} \Rbar.}
    \vspace{0.1cm} \\
    6. E superior aos querubins vos assentais.
    \hspace{\fill}{\color{red} \Rbar.}
    \vspace{0.1cm} \\
    7. Sede bendito no celeste firmamento.
    \hspace{\fill}{\color{red} \Rbar.}
    \vspace{0.2cm} \\
    \textcolor{red}{Oração}
    \vspace{0.1cm} \\
    Oremos.
    \vspace{0.1cm}\\
    Ó Deus, \\
    entre clarões de fogo, \\
    destes na montanha do Sinai \\
    a antiga lei a Moisés \\
    e hoje manifestastes a nova aliança \\
    no fogo do Espírito Santo. \\
    Concedei que ardamos sempre \\
    pelo mesmo Espírito \\
    que, de modo admirável, \\
    infundistes nos vossos Apóstolos, \\
    e o novo Israel, congregado de todos os povos, \\
    acolha com alegria o mandamento supremo \\
    do vosso amor. \\
    Por Cristo, nosso Senhor. \\
    {\color{red} \Rbar.} Amém.
    \vspace{0.2cm} \\

\end{flushleft}

\begin{center}

    \textcolor{red}{Terceira Leitura}

\end{center}

\begin{flushright}
    \textit{Ossos ressequidos, vou fazer entrar um espírito em vós, \\ e voltareis à vida.}
\end{flushright}

\begin{flushleft}

    \vspace{0.2cm}
    Leitura da Profecia de Ezequiel
    \hspace{\fill}
    \textcolor{red}{37,1--14}
    \vspace{0.2cm} \\
    Naqueles dias, \\
    a mão do Senhor estava sobre mim \\
    e por seu espírito ele me levou para fora \\
    e me deixou no meio de uma planície cheia de ossos \\
    e me fez andar no meio deles em todas as direções. \\
    Havia muitíssimo ossos na planície \\
    e estavam ressequidos. \\
    Ele me perguntou: \\
    ``Filho do homem, será que estes ossos podem voltar à vida?'' \\
    e eu respondi: ``Senhor Deus, só tu o sabes''. \\
    e ele me disse: \\
    ``Profetiza sobre estes ossos e dize: \\
    `Ossos ressequidos, escutai a palavra do Senhor!' \\
    Assim diz o Senhor Deus a estes ossos: \\
    Eu mesmo vou fazer entrar um espírito em vós \\
    e voltareis à vida. \\
    Porei nervos em vós, farei crescer carne \\
    e estenderei a pele por cima. \\
    Porei em vós um espírito, para que possais voltar à vida. \\
    Assim sabereis que eu sou o Senhor''. \\
    Profetizei como foi ordenado. \\
    Enquanto eu profetizava, \\
    ouviu-se primeiro um rumor, e logo um estrondo, \\
    quando os ossos se aproximaram uns dos outros. \\
    Olhei e vi nervos e carne crescendo sobre os ossos \\
    e, por cima, a pele que se estendia. \\
    Mas não tinham nenhum sopro de vida. \\
    Ele me disse: \\
    ``Profetiza para o espírito, profetiza, filho do homem! \\
    Dirás ao espírito: `Assim diz o Senhor Deus: \\
    Vem dos quatro ventos, ó espírito, \\
    vem soprar sobre estes mortos, \\
    para que eles possam voltar à vida'\,''. \\
    Profetizei como me foi ordenado, \\
    e o espírito entrou neles. \\
    Eles voltaram à vida e puseram-se de pé: \\
    era uma imensa multidão! \\
    Então ele me disse: \\
    ``Filho do homem, estes ossos são toda a casa de Israel. \\
    É isto que eles dizem: \\
    `Nossos ossos estão secos, nossa esperança acabou, \\
    estamos perdidos!' \\
    Por isso, profetiza e dize-lhes: \\
    `Assim fala o Senhor Deus: \\
    Ó meu povo, vou abrir as vossas sepulturas \\
    e conduzir-vos para a terra de Israel; \\
    e quando eu abrir as vossas sepulturas \\
    e vos fizer sair delas, sabereis que eu sou o Senhor. \\
    Porei em vós o meu espírito, para que vivais \\
    e vos colocarei em vossa terra. \\
    Então sabereis que eu, o Senhor, digo e faço, \\
    --oráculo do Senhor--'\,''.
    \vspace{0.1cm} \\
    Palavra do Senhor \\
    {\color{red} \Rbar.} Graças a Deus.
    \vspace{0.2cm} \\
    \textcolor{red}{Salmo Responsorial
        \hspace{\fill} Sl 106(107),2--3.4--5.6--7.8--9}
    \vspace{0.1cm} \\
    {\color{red} \Rbar.} Dai graças ao Senhor, porque ele é bom, porque eterna é a sua misericórdia!
    \hspace{\fill}
    {\color{red} (\Rbar. 1)}
    \vspace{0.1cm} \\
    1. Que o digam os libertos do Senhor, \\
    que da mão dos opressores os salvou \\
    e de todas as nações os reuniu, \\
    do Oriente, Ocidente, Norte e Sul.
    \hspace{\fill}{\color{red} \Rbar.}
    \vspace{0.1cm} \\
    2. Uns vagavam no deserto, extraviados, \\
    sem acharem o caminho da cidade. \\
    Sofriam fome e também sofriam sede,
    e sua vida ia aos poucos definhando.
    \hspace{\fill}{\color{red} \Rbar.}
    \vspace{0.1cm} \\
    3. Mas gritaram ao Senhor na aflição, \\
    e ele os libertou daquela angústia. \\
    Pelo caminho vem seguro os conduziu \\
    para chegaram à cidade onde morar.
    \hspace{\fill}{\color{red} \Rbar.}
    \vspace{0.1cm} \\
    4. Agradeçam ao Senhor por seu amor \\
    e por suas maravilhas entre os homens! \\
    Deu de beber aos que sofriam tanta sede \\
    e os famintos saciou com muitos bens!.
    \hspace{\fill}{\color{red} \Rbar.}
    \vspace{0.2cm} \\
    \textcolor{red}{Oração}
    \vspace{0.1cm} \\
    Oremos.
    \vspace{0.1cm}\\
    Senhor, Deus todo-poderoso, \\
    que restaurais tudo o que decaiu \\
    e o conservais restaurado, \\
    fazei crescer os povos \\
    a serem renovados pela glorificação do vosso nome, \\
    para que todos os purificados pelo santo batismo \\
    sejam dirigidos sempre pela vossa inspiração. \\
    Por Cristo, nosso Senhor. \\
    {\color{red} \Rbar.} Amém.

\end{flushleft}

\newpage

\begin{center}

    \textcolor{red}{Quarta Leitura}

\end{center}

\begin{flushright}
    \textit{Sobre meus servos e servas derramarei o meu espírito.}
\end{flushright}

\begin{flushleft}

    \vspace{0.2cm}
    Leitura da Profecia de Joel
    \hspace{\fill}
    \textcolor{red}{3,1--5}
    \vspace{0.2cm} \\
    Assim diz o Senhor: \\
    ``Derramarei o meu espírito sobre todo ser humano, \\
    e vossos filhos e filhas profetizarão, \\
    vossos anciãos terão sonhos \\
    e vossos jovens terão visões; \\
    também sobre meus servos e servas, \\
    naqueles dias, derramarei o meu espírito. \\
    Colocarei sinais no céu e na terra, \\
    sangue, fogo e rolos de fumaça; \\
    o sol se transformará em trevas \\
    e a lua, em sangue, antes de chegar o dia do Senhor, \\
    dia grandioso e terrível. \\
    Então, todo aquele que invocar o nome do Senhor, \\
    será salvo, pois, no monte Sião e em Jerusalém, \\
    haverá salvação, como que o Senhor chamar''.
    \vspace{0.1cm} \\
    Palavra do Senhor \\
    {\color{red} \Rbar.} Graças a Deus.
    \vspace{0.2cm} \\
    \textcolor{red}{Salmo Responsorial
        \hspace{\fill} Sl 103(104),1--2a,24.35c.27--29bc--30}
    \vspace{0.1cm} \\
    {\color{red} \Rbar.} Enviai o vosso Espírito, Senhor, e da terra toda a face renovai.
    \hspace{\fill}
    {\color{red} (\Rbar. 30)}
    \vspace{0.1cm} \\
    1. Bendize, ó minha alma, ao Senhor! \\
    Ó meu Deus e meu Senhor, como és tão grande! \\
    De majestade e esplendor vos revestis \\
    e de luz vos envolveis como num manto.
    \hspace{\fill}{\color{red} \Rbar.}
    \vspace{0.1cm} \\
    2. Quão numerosas, ó Senhor, são vossas obras, \\
    e que sabedoria em todas elas! \\
    Encheu-se a terra com as vossas criaturas. \\
    Bendize, ó minha alma, ao Senhor!
    \hspace{\fill}{\color{red} \Rbar.}
    \vspace{0.1cm} \\
    3. Todos eles, ó Senhor, de vós esperam \\
    que a seu tempo vós lhes deis o alimento; \\
    vós lhes dais o que comer e eles recolhem, \\
    vós abris a vossa mão e eles se fartam.
    \hspace{\fill}{\color{red} \Rbar.}
    \vspace{0.1cm} \\
    4. Se tirais o seu respiro, eles perecem \\
    e voltam para pó de onde vieram; \\
    enviais o vosso espírito e renascem \\
    e da terra toda a face renovais.
    \hspace{\fill}{\color{red} \Rbar.}
    \vspace{0.2cm} \\
    \textcolor{red}{Oração}
    \vspace{0.1cm} \\
    Oremos.
    \vspace{0.1cm}\\
    Senhor, realizai em nós a vossa promessa, \\
    para que a vinda do Espírito Santo \\
    nos transforme em testemunhas \\
    do Evangelho de nosso Senhor Jesus Cristo \\
    diante do mundo. \\
    Por Cristo, nosso Senhor. \\
    {\color{red} \Rbar.} Amém.

    \newpage

    \textcolor{red}{Glória}
    \vspace{0.2cm} \\
    Glória a Deus nas alturas, \\
    e paz na terra aos homens por Ele amados. \\
    Senhor Deus, rei dos céus, \\
    Deus Pai todo-poderoso: \\
    nós vos louvamos, \\
    nós vos adoramos, \\
    nós vos glorificamos, \\
    nós vos damos graça \\
    por vossa imensa glória. \\
    Senhor Jesus Cristo, Filho Unigênito, \\
    Senhor Deus, Cordeiro de Deus, \\
    Filho de Deus Pai. \\
    Vós que tirais o pecado do mundo, \\
    tende piedade de nós. \\
    Vós que tirais o pecado do mundo, \\
    acolhei a nossa súplica. \\
    Vós que estais à direita do Pai, \\
    tende piedade de nós. \\
    Só vós sois o Santo, \\
    só vós, o Senhor, \\
    só vós, o Altissimo, \\
    Jesus Cristo, \\
    com o Espírito Santo, \\
    na glória de Deus Pai. \\
    Amém.
    \vspace{0.2cm} \\
    \textcolor{red}{Oração}
    \vspace{0.1cm} \\
    Oremos.
    \vspace{0.1cm}\\
    Deus eterno e todo-poderoso, \\
    quisestes que o mistério pascal \\
    se completasse durante cinquenta dias, \\
    até a vinda do Espírito Santo. \\
    Fazei que todas as nações dispersas pela terra, \\
    na diversidade de suas línguas, \\
    se unam no louvor do vosso nome. \\
    Por nosso Senhor Jesus Cristo, vosso Filho, \\
    na unidade do Espírito Santo.\\
    {\color{red} \Rbar.} Amém.

\end{flushleft}

\begin{center}

    \textcolor{red}{Quinta Leitura}

\end{center}

\begin{flushright}
    \textit{O Espírito intercede em nosso favor \\ com gemidos inefáveis.}
\end{flushright}

\begin{flushleft}

    \vspace{0.2cm}
    Leitura da Carta de São Paulo aos Romanos
    \hspace{\fill}
    \textcolor{red}{8,22--27}
    \vspace{0.2cm} \\
    Irmãos: \\
    Sabemos que toda a criação, até ao tempo presente, \\
    está gemendo como que em dores de parto, \\
    e não somente ela, mas nós também, \\
    que temos os primeiros frutos do Espíritos, \\
    estamos interiormente gemendo, \\
    aguardando a adoção filial \\
    e a libertação para o nosso corpo. \\
    Pois já fomos salvos, mas na esperança. \\
    Ora, o objeto da esperança \\
    não é aquilo que a gente está vendo; \\
    como pode alguém esperar o que já vê? \\
    Mas, se esperamos o que não vamos, \\
    é porque o estamos aguardando mediante a perseverança. \\
    Também o Espírito vem em socorro da nossa fraqueza. \\
    Pois nós não sabemos o que pedir, nem como pedir; \\
    é o próprio Espírito que intercede em nosso favor, \\
    com gemidos inefáveis. \\
    E aquele que penetra o íntimo dos corações \\
    sabe qual é a intenção do Espírito. \\
    Pois é sempre segundo Deus \\
    que o Espírito intercede em favor dos santos.
    \vspace{0.1cm} \\
    Palavra do Senhor \\
    {\color{red} \Rbar.} Graças a Deus.
    \vspace{0.2cm} \\
    \textcolor{red}{Aclamação ao Evangelho}
    \vspace{0.1cm} \\
    {\color{red} \Rbar.} Aleluia, Aleluia, Aleluia. \\
    {\color{red} \Vbar.} Vinde, Espírito Divino, \\
    e enchei com vossos dons os corações dos fiéis; \\
    e acendei neles o amor como um fogo abrasador!
    \hspace{\fill}{\color{red} \Rbar.}

\end{flushleft}

\begin{center}

    \textcolor{red}{Evangelho}

\end{center}

\begin{flushright}
    \textit{Jorrarão rios de água viva.}
\end{flushright}

\begin{flushleft}

    \vspace{0.2cm}
    {\color{red} \Vbar.} O Senhor esteja convosco. \\
    {\color{red} \Rbar.} Ele está no meio de nós.
    \vspace{0.1cm} \\
    {\color{red} \grecross} Proclamação do Evangelho de Jesus Cristo, segundo João.
    \hspace{\fill}
    \textcolor{red}{7,37--39} \\
    {\color{red} \Rbar.} Glória a vós, Senhor.
    \vspace{0.2cm} \\
    No último dia da festa, o dia mais solene, \\
    Jesus, em pé, proclamou em voz alta: \\
    ``Se alguém tem sede, venha a mim, e beba. \\
    Aquele que crê em im, conforme diz a Escritura, \\
    rios de água viva jorrarão do seu interior''. \\
    Jesus falava do Espírito, \\
    que deviam receber os que tivessem fé nele; \\
    pois ainda não tinha sido dado o Espírito, \\
    porque Jesus ainda não tinha sido glorificado.
    \vspace{0.1cm} \\
    Palavra da Salvação. \\
    {\color{red} \Rbar.} Glória a vós, Senhor.
    \vspace{0.2cm} \\
    \textcolor{red}{Credo --- Símbolo Apostólico}
    \vspace{0.2cm} \\
    Creio em Deus Pai todo-poderoso, \\
    criador do céu e da terra. \\
    E em Jesus Cristo, seu único Filho, nosso Senhor, \\
    que foi concebido pelo poder do Espírito Santo; \\
    nasceu da virgem Maria; \\
    padeceu sob Pôncio Pilatos, \\
    foi crucificado, morto e sepultado. \\
    Desceu à mansão dos mortos; \\
    ressuscitou ao terceiro dia, \\
    subiu aos céus; \\
    está sentado à direita de Deus Pai todo-poderoso, \\
    donde há de vir a julgar os vivos e os mortos. \\
    Creio no Espírito Santo; \\
    na Santa Igreja católica; \\
    na comunhão dos santos; \\
    na remissão dos pecados; \\
    na ressurreição da carne; \\
    na vida eterna. \\
    {\color{red} \Rbar.} Amém.

    \newpage

    \textcolor{red}{Oração Universal}
    \vspace{0.2cm} \\
    \lettrine[findent=2pt]{\color{red}I}{rmãos} e irmãs em Cristo:
    \newline
    Oremos ao Senhor do universo
    \newline
    para que envie de novo o seu Espírito
    \newline
    sobre a Igreja e sobre o mundo,
    \newline
    dizendo \textcolor{red}{(ou:} cantando\textcolor{red}{)}, com alegria:
    \vspace{0.1cm}
    \newline
    {\color{red} \Rbar.} Mandai, Senhor, o vosso Espírito.
    \newline
    \textcolor{red}{ou:} Enviai, Senhor, o vosso Espírito.
    \newline
    \textcolor{red}{ou:} Desça, Senhor, o vosso Espírito.
    \vspace{0.1cm}
    \newline
    {\color{red} 1.} Sobre as Igrejas que procuram a unidade.
    \vspace{0.1cm}
    \newline
    {\color{red} 2.} Sobre o Papa Francisco, sobre os bispos, os presbíteros e os diáconos.
    \vspace{0.1cm}
    \newline
    {\color{red} 3.} Sobre as monjas e os monges contemplativos.
    \vspace{0.1cm}
    \newline
    {\color{red} 4.} Sobre as religiosas e os religiosos.
    \vspace{0.1cm}
    \newline
    {\color{red} 5.} Sobre os que trabalham em terras de missão.
    \vspace{0.1cm}
    \newline
    {\color{red} 6.} Sobre as jovens e os jovens inquietos pelo futuro.
    \vspace{0.1cm}
    \newline
    {\color{red} 7.} Sobre as crianças que comungam pela primeira vez e os catequistas.
    \vspace{0.1cm}
    \newline
    {\color{red} 8.} Sobre os esposos que se amam e sobre os que deixaram de se amar.
    \vspace{0.1cm}
    \newline
    {\color{red} 9.} Sobre os doentes, sobre os que choram e sobre os moribundos.
    \vspace{0.1cm}
    \newline
    {\color{red} 10.} Sobre os fiéis das nossas comunidades paroquiais.
    \vspace{0.1cm} \\
    \lettrine[findent=2pt]{\color{red}D}{eus} eterno e omnipotente,
    \newline
    que enviais aos corações dos vossos filhos
    \newline
    o Espírito Santo do Pentecostes,
    \newline
    tornai-nos suas testemunhas,
    \newline
    para proclamarmos as vossas maravilhas.
    \newline
    Por Cristo, nosso Senhor.
    \newline
    {\color{red} \Rbar.} Amém.

\end{flushleft}

\end{document}

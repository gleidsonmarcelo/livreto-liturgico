\documentclass{book}
\usepackage{import}
\import{/home/vatican/livreto-liturgico/Hebdmada-Quinta-In-Quadragesima/Feria-Quinta/modules/}{def-packages}
\import{/home/vatican/livreto-liturgico/Hebdmada-Quinta-In-Quadragesima/Feria-Quinta/modules/}{def-macros}
\import{/home/vatican/livreto-liturgico/Hebdmada-Quinta-In-Quadragesima/Feria-Quinta/modules/}{def-pages}
\geometry{a5paper,hdivide={1cm,*,1cm},vdivide={1cm,*,1cm}}
\begin{document}
\begin{center}
    \LARGE Arquidiocese de Olinda e Recife
    \vspace{.2cm} \\
    \Large Paróquia Nossa Senhora dos Prazeres dos Maranguapes
    \vspace{5cm} \\
    \textcolor{VioletRed2}{\huge Quinta-Feira da V Semana da Quaresma}
    \vspace{5cm} \\
    \Large Santa Missa Presidida pelo Ex.mo Rev.mo
    \vspace{.2cm} \\
    \textcolor{VioletRed2}{\huge Dom Paulo Jackson}
    \vspace{.2cm} \\
    \Large Arcebispo de Olinda e Recife
    \vspace{.5cm} \\
    \Large Posse do pároco Rev.mo Pe. André Canuto
    \vspace{.2cm}
    \vspace{\fill}\\
    \Large Paulista, 21 de Março de 2024
\end{center}
\cleardoublepage{}
\begin{center}
    \textbf{Ritos Iniciais}
\end{center}
\begin{flushleft}
    \textcolor{VioletRed2}{Cento de Entrada}
    \vspace{.2cm} \\
    \textbf{Eis o tempo de conversão! Eis o dia da salvação! Ao Pai voltemos, juntos andemos: Eis o tempo de conversão!}
    \vspace{.2cm} \\
    1. Os caminhos do Senhor são verdade, são amor: Dirigi os passos meus, em vós espero, oh Senhor! Ele guia ao bom caminho, quem errou e quer voltar. Ele é bom, fiel e justo, Ele busca e vem salvar. \\
    2. Viverei com o Senhor, Ele é o meu sustento. Eu confio, mesmo quando minha dor não mais aguento. Têm valor aos olhos Seus meu sofrer e meu morrer: Libertai o vosso servo e fazei-o reviver! \\
    3. A Palavra do Senhor é a luz do meu caminho; ela é vida, é alegria: vou guardá-la com carinho. Sua lei, seu mandamento é viver a caridade: caminhemos todos juntos, construindo a unidade!
    \vspace{.2cm} \\
    \textcolor{VioletRed2}{Saudação}
    \vspace{.2cm}\\
    Em nome do Pai e do Filho e do Espírito Santo. \\
    {\color{VioletRed2} \Rbar.} Amém. \\
    A paz esteja convosco. \\
    {\color{VioletRed2} \Rbar.} Bendito seja Deus \\
    que nos reuniu no amor de Cristo.
    \vspace{.2cm} \\
    \textcolor{VioletRed2}{Leitura do Decreto de Nomeação do Novo Pároco}
    \vspace{.2cm} \\
    \textcolor{VioletRed2}{Ato Penitencial}
    \vspace{.2cm} \\
    Em Jesus Cristo, o Justo, \\
    que intercede por nós e nos reconcilia com o Pai, \\
    abramos o nosso espírito ao arrependimento \\
    para sermos dignos de nos aproximar da mesa do Senhor.
    \vspace{.2cm} \\
    1. Senhor, que sois a plenitude da verdade e da graça, tende piedade de nós.
    \vspace{.2cm} \\
    \textbf{KYRIE, KYRIE KYRIE, ELEISON! (bis)}
    \vspace{.2cm} \\
    2. Cristo, que vos tornastes pobre para nos enriquecer, tende piedade de nós.
    \vspace{.2cm} \\
    \textbf{CHISTE, CHISTE, CHISTE, ELEISON! (bis)}
    \vspace{.2cm} \\
    3. Senhor, que viestes para fazer de nós o vosso povo santo, tende piedade de nós.
    \vspace{.2cm} \\
    \textbf{KYRIE, KYRIE KYRIE, ELEISON! (bis)}
    \vspace{.2cm} \\
    Deus todo-poderoso tenha compaixão de nós, \\
    perdoe os nossos pecados \\
    e nos conduz à vida eterna. \\
    {\color{VioletRed2} \Rbar.} Amém.
    \newpage
    \textcolor{VioletRed2}{Coleta}
    \vspace{.2cm} \\
    Assisti, Senhor, aqueles que vos suplicam \\
    e guardai com solicitude \\
    os que esperam em vossa misericórdia, \\
    para que, purificados dos seus pecados, \\
    levem uma vida santa e mereçam  \\
    torna-se herdeiros das vossas promessas. \\
    Por nosso Senhor Jesus Cristo, vosso Filho, que é Deus, \\
    e convosco vive e reina, na unidade do Espírito Santo, \\
    por todos os séculos dos séculos. \\
    {\color{VioletRed2} \Rbar.} Amém. \\
\end{flushleft}
\begin{center}
    \textbf{Liturgia da Palavra}
    \vspace{.2cm}\\
    \textcolor{VioletRed2}{Primeira Leitura}
\end{center}
\begin{flushright}
    \textit{Farei de ti pai de uma multidão de nações.}
\end{flushright}
\begin{flushleft}
    Leitura do Livro do Gênesis
    \hspace{\fill}
    \textcolor{VioletRed2}{17,3-9}
    \vspace{.2cm} \\
    Naqueles dias, \\
    Abrão prostrou-se com o rosto por terra. \\
    E Deus lhe disse: \\
    ``Eis a minha aliança contigo: \\
    tu serás pai de uma multidão de nações. \\
    Já não te chamarás Abrão, \\
    mas o teu nome será Abraão, \\
    porque farei de ti o pai de uma multidão de nações. \\
    Farei crescer tua descendência infinitamente. \\
    Farei nascer de ti nações, \\
    e reis sairão de ti. \\
    Estabelecerei minha aliança entre mim e ti \\
    e teus descendentes para sempre; \\
    uma aliança eterna, \\
    para que eu seja teu Deus \\
    e o Deus de teus descendentes. \\
    A ti e aos teus descendentes \\
    darei a terra em que vives como estrangeiro, \\
    todo o país de Canaã como propriedade para sempre. \\
    E eu serei o Deus dos teus descendentes''. \\
    Deus disse a Abraão: \\
    ``Guarda a minha aliança, \\
    tu e a tua descendência para sempre''.
    \vspace{.2cm} \\
    Palavra do Senhor \\
    {\color{VioletRed2} \Rbar.} Graças a Deus.
    \newpage
    \textcolor{VioletRed2}{Salmo Responsorial
        \hspace{\fill} Sl 104(105),4-5.6-7.8-9 (\RbarRed{} 8a)}
    \vspace{.2cm} \\
    {\color{VioletRed2} \Rbar.} O Senhor se lembra sempre da Aliança!
    \vspace{.2cm} \\
    Procurai o Senhor Deus e seu poder, \textsuperscript{\gresixstar{}} \\
    buscai constantemente a sua face! \\
    Lembrai as maravilhas que ele fez, \textsuperscript{\gresixstar{}} \\
    seus prodígios e as palavras de seus lábios!
    \hspace{\fill}{\color{VioletRed2} \Rbar.}
    \vspace{.2cm} \\
    Descendentes de Abraão, seu servidor, \textsuperscript{\gresixstar{}} \\
    e filhos de Jacó, seu escolhido, \\
    ele mesmo, o Senhor, é nosso Deus, \textsuperscript{\gresixstar{}} \\
    vigoram suas leis em toda a terra.
    \hspace{\fill}{\color{VioletRed2} \Rbar.}
    \vspace{.2cm} \\
    Ele sempre se recorda da Aliança, \textsuperscript{\gresixstar{}} \\
    promulgada a incontáveis gerações; \\
    da Aliança que ele fez com Abraão, \textsuperscript{\gresixstar{}} \\
    e do seu santo juramento a Isaac.
    \hspace{\fill}{\color{VioletRed2} \Rbar.}
    \vspace{.2cm} \\
    \textcolor{VioletRed2}{Aclamação ao Evangelho}
    \vspace{.2cm} \\
    {\color{VioletRed2} \Rbar.} Glória a Cristo, Palavra eterna do Pai, que é amor! \\
    {\color{VioletRed2} \Vbar.} Oxalá ouvísseis hoje a sua voz. \\
    Não fecheis os corações como em Meriba!
    \hspace{\fill}{\color{VioletRed2} \Rbar.}
\end{flushleft}
\begin{center}
    \textcolor{VioletRed2}{Evangelho}
\end{center}
\begin{flushright}
    \textit{Vosso pai Abraão exultou, por ver o meu dia.}
\end{flushright}
\begin{flushleft}
    {\color{VioletRed2} \Vbar.} O Senhor esteja convosco. \\
    {\color{VioletRed2} \Rbar.} Ele está no meio de nós.
    \vspace{.2cm} \\
    {\color{VioletRed2} \grecross} Proclamação do Evangelho de Jesus Cristo, segundo João.
    \hspace{\fill}
    \textcolor{VioletRed2}{8,51-59} \\
    {\color{VioletRed2} \Rbar.} Glória a vós, Senhor.
    \vspace{.2cm} \\
    Naquele tempo, disse Jesus aos judeus: \\
    ``Em verdade, em verdade, eu vos digo: \\
    se alguém guardar a minha palavra, \\
    jamais verá a morte''. \\
    Disseram então os judeus: \\
    ``Agora sabemos que tens um demônio. \\
    Abraão morreu e os profetas também, \\
    e tu dizes: \\
    `Se alguém guardar a minha palavra \\
    jamais verá a morte'. \\
    Acaso és maior do que nosso pai Abraão, \\
    que morreu, como também os profetas? \\
    Quem pretendes tu ser?''. \\
    Jesus respondeu: \\
    ``Se me glorifico a mim mesmo, \\
    minha glória não vale nada. \\
    Quem me glorifica é o meu Pai, \\
    aquele que vós dizeis ser o vosso Deus. \\
    No entanto, não o conheceis. \\
    Mas eu o conheço \\
    e, se dissesse que não o conheço, \\
    seria um mentiroso, como vós! \\
    Mas eu o conheço e guardo a sua palavra. \\
    Vosso pai Abraão exultou, \\
    por ver o meu dia; \\
    ele o viu, e alegrou-se''. \\
    Os judeus disseram-lhe então: \\
    ``Nem sequer cinquenta anos tens, \\
    e viste Abraão!?'' \\
    Jesus respondeu: \\
    ``Em verdade, em verdade vos digo, \\
    antes que Abraão existisse, \\
    eu sou''. \\
    Então eles pegaram em pedras para apedrejar Jesus, \\
    mas ele escondeu-se e saiu do Templo. \\
    Palavra da Salvação.
    \vspace{.2cm} \\
    Palavra da Salvação. \\
    {\color{VioletRed2} \Rbar.} Glória a vós, Senhor.
    \vspace{.2cm} \\
    \textcolor{VioletRed2}{Oração Universal}
    \vspace{.2cm} \\
    \lettrine[findent=2pt]{\color{VioletRed2}I}{rmãos} e irmãs:
    \newline
    Abraão é o nosso pai na fé,
    \newline
    mas Jesus é a Palavra que dá vida.
    \newline
    Oremos ao Deus da Aliança, dizendo:
    \vspace{.2cm}
    \newline
    {\color{VioletRed2} \Rbar.} Ouvi, Senhor, a nossa prece.
    \vspace{.2cm}
    \newline
    {\color{VioletRed2} 1.} Pelo nosso Arcebispo, Dom Paulo Jackson,
    \newline
    para que que Deus o fortaleça na fé
    \newline
    para cumprir sua missão de pastor,
    \newline
    oremos.
    \vspace{.2cm}
    \newline
    {\color{VioletRed2} 2.} Pelo nosso novo pároco, padre André,
    \newline
    para que que nesta nova missão
    \newline
    possa receber de Deus as graças necessárias de Deus,
    \newline
    oremos.
    \vspace{.2cm}
    \newline
    {\color{VioletRed2} 3.} Pelos cristãos que ouvem a palavra de Jesus,
    \newline
    pelos que a guardam fielmente até a morte
    \newline
    e pelos que a fazem frutificar em boas obras,
    \newline
    oremos.
    \vspace{.2cm}
    \newline
    {\color{VioletRed2} 4.} Pelos descendentes de Abraão, o pai dos crentes,
    \newline
    pelas nações que nele tiveram a sua origem
    \newline
    e pelos que moram na terra que Deus lhes deu,
    \newline
    oremos.
    \vspace{.2cm}
    \newline
    {\color{VioletRed2} 5.} Pelos que gostam ded ouvir Jesus falar do Pai,
    \newline
    pelos que creem que Ele está vivo para sempre
    \newline
    e pelos que se alegram com os dons que d'Ele receberam,
    \newline
    oremos.
    \vspace{.2cm}
    \newline
    {\color{VioletRed2} 6.} Pelos homens com que Deus faz aliança ainda hoje,
    \newline
    pelos que a mantêm e a vivem com alegria
    \newline
    e por aqueles que a rejeitam e abandonam,
    \newline
    oremos.
    \vspace{.2cm}
    \newline
    {\color{VioletRed2} 7.} Por esta assembleia e pelos outros paroquianos,
    \newline
    pelos mais novos, pelos mais velhos e pelos idosos
    \newline
    e por aqueles que estão prestes a deixar-nos,
    \newline
    oremos.
    \vspace{.2cm} \\
    \lettrine[findent=2pt]{\color{VioletRed2}S}{enhor}, fazei-nos viver sempre na Aliança
    \newline
    que um dia estabelecestes com Abraão
    \newline
    e cumpristes plenamente em vosso Filho.
    \newline
    Ele que é Deus convosco na unidade do Espírito Santo.
    \newline
    {\color{VioletRed2} \Rbar.} Amém.
\end{flushleft}
\begin{center}
    \textbf{Liturgia Eucarística}
\end{center}
\begin{flushleft}
    \textcolor{VioletRed2}{Apresentação das oferendas}
    \vspace{.2cm} \\
    1. Sê bendito, Senhor, para sempre \\
    Pelos frutos das nossas jornadas! \\
    Repartidos na mesa do reino \\
    Anunciam a paz almejada!
    \vspace{.2cm} \\
    \textbf{Senhor da vida \\
        Tu és a nossa salvação! \\
        Ao prepararmos a tua mesa \\
        Em ti buscamos ressurreição!}
    \vspace{.2cm} \\
    2. Sê bendito, Senhor, para sempre \\
    Pelos mares, os rios e as fontes! \\
    Nos recordam a tua justiça \\
    Que nos leva a um novo horizonte! \\
    3. Sê bendito, Senhor, para sempre \\
    Pelas bênçãos qual chuva torrente! \\
    Tu fecundas o chão desta vida \\
    Que abriga uma nova semente
    \newpage
    \textcolor{VioletRed2}{Preparação para oferendas}
    \vspace{.2cm} \\
    Orai, irmãos e irmãs, \\
    para que o sacrifício da Igreja, \\
    nesta pausa restauradora na caminhada rumo ao céu, \\
    seja aceito por Deus Pai todo-poderoso.
    \vspace{.2cm} \\
    Receba o Senhor por tuas mãos este sacrifício, \\
    para glória do seu nome, \\
    para nosso bem \\
    e de toda a santa Igreja.
    \vspace{.2cm} \\
    {\color{VioletRed2} \Rbar.} Amém.
    \vspace{.2cm} \\
    \textcolor{VioletRed2}{Sobre as oferendas}
    \vspace{.2cm} \\
    Acolhei, Senhor, com bondade, este sacrifício, \\
    para que seja proveitoso à nossa conversão \\
    e à salvação do mundo inteiro. \\
    Por Cristo, nosso Senhor. \\
    {\color{VioletRed2} \Rbar.} Amém.
\end{flushleft}
\begin{center}
    \textcolor{VioletRed2}{Oração Eucarística III \\ \small Prefácio da paixão do Senhor I}
\end{center}
\begin{flushleft}
    {\color{VioletRed2} \Vbar.} O Senhor esteja convosco. \\
    {\color{VioletRed2} \Rbar.} Ele está no meio de nós. \\
    {\color{VioletRed2} \Vbar.} Corações ao alto. \\
    {\color{VioletRed2} \Rbar.} O nosso coração está em Deus. \\
    {\color{VioletRed2} \Vbar.} Demos graças ao Senhor, nosso Deus. \\
    {\color{VioletRed2} \Rbar.} É o nosso dever e nossa salvação.
    \vspace{.2cm} \\
    Na Verdade, é digno e justo, \\
    é nosso dever e salvação dar-vos graças, \\
    sempre e em todo o lugar, \\
    Senhor, Pai santo, \\
    Deus eterno e todo-poderoso.
    \vspace{.2cm} \\
    Pois, pela paixão salvadora do vosso Filho, \\
    o mundo inteiro recebeu \\
    a missão de proclamar a vossa glória. \\
    A força radiante da cruz \\
    manifesta o julgamento do mundo \\
    e o poder de Jesus Crucificado.
    \vspace{.2cm} \\
    Por isso, Senhor, \\
    também nós, com todos os Anjos e Santos, \\
    vos aclamamos, \\
    cantando {\color{VioletRed2}(}dizendo{\color{VioletRed2})} alegres a uma só voz:
    \newpage
    Santo, Santo, Santo, \\
    Senhor, Deus do Universo! \\
    O Céu e a terra proclamam a vossa glória. \\
    Hosana nas alturas! \\
    Bendito o que vem \\
    em nome do Senhor! \\
    Hosana nas alturas!
    \vspace{.2cm} \\
    {\color{VioletRed2}CP} Na verdade, vós sois Santo, ó Deus do universo, \\
    e tudo o que criastes proclama o vosso louvor, \\
    porque, por Jesus Cristo, vosso Filho e Senhor nosso, \\
    e pela força do Espírito Santo, \\
    dais vida e santidade a todas as coisas \\
    e não cessais de reunir para vós um povo \\
    que vos ofereça em toda parte, \\
    do nascer ao pôr do sol, um sacrifício perfeito.
    \vspace{.2cm} \\
    {\color{VioletRed2}CC} Por isso, ó Pai, nós vos suplicamos: \\
    santificai pelo Espírito Santo \\
    as oferendas que vos apresentamos \\
    para serem consagradas \\
    a fim de que se tornam \\
    o Corpo e \grecrossRed{} o Sangue de vosso Filho, \\
    nosso Senhor Jesus Cristo, \\
    que nos mandou celebrar estes mistérios.
    \vspace{.2cm} \\
    {\color{VioletRed2} \Rbar.} Enviai o vosso Espírito Santo!
    \vspace{.2cm} \\
    Na noite em que ia ser entregue, \\
    Jesus tomou o pão \\
    pronunciou a bênção da ação de graças, \\
    partiu e o deu a seus discípulos, \\
    dizendo:
    \vspace{.2cm} \\
    TOMAI, TODOS, E COMEI: \\
    ISTO É O MEU CORPO, \\
    QUE SERÁ ENTREGUE POR VÓS.
    \vspace{.2cm} \\
    Do mesmo modo, \\
    ao fim da ceia, \\
    ele tomou o cálice em suas mãos, \\
    pronunciou a bênção da ação de graças, \\
    e o deu a seus discípulos, \\
    dizendo:
    \newpage
    TOMAI, TODOS E BEBEI: \\
    ESTE É O CÁLICE DO MEU SANGUE, \\
    O SANGUE DA NOVA E ETERNA ALIANÇA, \\
    QUE SERÁ DERRAMADO POR VÓS E POR TODOS \\
    PARA REMISSÃO DOS PECADOS. \\
    FAZEI ISTO EM MEMÓRIA DE MIM.
    \vspace{.2cm} \\
    Eis o mistério dda fé!
    \vspace{.2cm} \\
    {\color{VioletRed2} \Rbar.} Anunciamos, Senhor, a vossa morte \\
    e proclamamos a vossa ressurreição. \\
    Vinde, Senhor Jesus!
    \vspace{.2cm} \\
    {\color{VioletRed2}CC} Celebrando agora, ó Pai, \\
    o memorial da paixão redentora do vosso Filho, \\
    da sua gloriosa ressurreição e ascensão ao céu, \\
    e enquanto esperamos sua nova vinda, \\
    nós vos oferecemos em ação de graças \\
    este sacrifício vivo e santo.
    \vspace{.2cm} \\
    {\color{VioletRed2} \Rbar.} Aceitai, ó Senhor, a nossa oferta!
    \vspace{.2cm} \\
    Olhai com bondade a oblação da vossa Igreja e \\
    reconhecei nela o sacrifício que nos reconciliou convosco; \\
    concedei que, alimentando-nos \\
    com o Corpo e o Sangue do vosso Filho, \\
    repletos do Espírito Santo, \\
    nos tornemos em Cristo \\
    um só corpo e um só espírito.
    \vspace{.2cm} \\
    {\color{VioletRed2} \Rbar.} O Espírito nos una num só corpo!
    \vspace{.2cm} \\
    {\color{VioletRed2}1C} Que o mesmo Espírito faça de nós uma eterna oferenda \\
    para alcançarmos a herança com os vossos eleitos: \\
    a santíssima Virgem Maria, Mãe de Deus, \\
    São José, seu esposo, \\
    os vossos santos Apóstolos e gloriosos Mártires, \\
    e todos os Santos, \\
    que não cessam de interceder por nós \\
    na vossa presença.
    \vspace{.2cm} \\
    {\color{VioletRed2} \Rbar.} Fazei de nós uma perfeita oferenda!
    \vspace{.2cm} \\
    {\color{VioletRed2}2C} Nós vos suplicamos, Senhor \\
    que este sacrifício da nossa reconciliação \\
    estenda a paz e a salvação ao mundo inteiro. \\
    Confirmai na fé e na caridade a vossa Igreja \\
    que caminha neste mundo \\
    com o vosso servo o Papa Francisco e o nosso Bispo Paulo Jackson, \\
    com os bispos do mundo inteiro, \\
    os presbíteros e diáconos, \\
    os outros ministros \\
    e o povo por vós redimido.
    \vspace{.2cm} \\
    Atendei propício às preces desta família, \\
    que reunistes em vossa presença. \\
    Reconduzi a vós, Pai de misericórdia, \\
    todos os vossos filhos e filhas \\
    dispersos pelo mundo inteiro.
    \vspace{.2cm} \\
    {\color{VioletRed2} \Rbar.} Lembrai-vos, ó Pai, da vossa Igreja!
    \vspace{.2cm} \\
    {\color{VioletRed2}3C} Acolhei com bondade no vosso reino \\
    os nossos irmãos e irmãs que partiram desta vida \\
    e todos os que morreram na vossa amizade. \\
    Unidos a eles, \\
    esperamos também nós \\
    saciar-nos eternamente da vossa glória,
    por Cristo, Senhor nosso.
    Por ele dais ao mundo \\
    todo bem e toda graça.
    \vspace{.2cm} \\
    {\color{VioletRed2}CP ou CC} Por Cristo, \\
    com Cristo, \\
    em Cristo, \\
    a vós, Deus Pai todo-poderoso, \\
    na unidade do Espírito Santo, \\
    toda a honra e toda a glória, \\
    por todos os séculos dos séculos.
    \vspace{.2cm} \\
    {\color{VioletRed2} \Rbar.} Amém.
\end{flushleft}
\begin{center}
    \textbf{Rito da Comunhão}
\end{center}
\begin{flushleft}
    Obedientes à palavra do Salvador \\
    e formados por seu divino ensinamento, \\
    ousamos dizer:
    \vspace{.2cm} \\
    Pai nosso que estais nos céus, \\
    santificado seja o vosso nome; \\
    venha a nós o vosso reino, \\
    seja feita a vossa vontade, \\
    assim na terra como no céu; \\
    o pão nosso de cada dia nos dai hoje; \\
    perdoai-nos as nossas ofensas, \\
    assim como nós perdoamos \\
    a quem nos tem ofendido; \\
    e não nos deixeis cair em tentação, \\
    mas livrai-nos do mal.
    \vspace{.2cm} \\
    Livrai-nos de todos os males, ó Pai, \\
    e dai-nos hoje a vossa paz. \\
    Ajudados pela vossa misericórdia, \\
    sejamos sempre livres do pecado \\
    e protegidos de todos os perigos, \\
    enquanto, aguardamos a feliz esperança, \\
    e a vinda do nosso Salvador, Jesus Cristo.
    \vspace{.2cm} \\
    {\color{VioletRed2} \Rbar.} Vosso é o reino, o poder e a glória para sempre!
    \vspace{.2cm} \\
    Senhor Jesus Cristo, \\
    dissestes aos vossos Apóstolos: \\
    Eu vos deixo a paz, eu vos dou a minha paz. \\
    Não olheis os nossos pecados, \\
    mas a fé que anima vossa Igreja; \\
    dai-lhe, segundo o vosso desejo, a paz e a unidade.
    \vspace{.2cm} \\
    Vós que sois Deus, com o Pai e o Espírito Santo.
    \vspace{.2cm} \\
    {\color{VioletRed2} \Rbar.} Amém.
    \vspace{.2cm} \\
    A paz do Senhor esteja sempre convosco.
    \vspace{.2cm} \\
    {\color{VioletRed2} \Rbar.} O amor de Cristo nos uniu.
    \vspace{.2cm} \\
    Em Jesus, que bos tornou todos irmãos e irmãs, \\
    saudai-vos com um sinal de reconciliação e de paz.
    \vspace{.2cm} \\
    Cordeiro de Deus, \\
    que tirais o pecado do mundo, \\
    tende piedade de nós. \\
    Cordeiro de Deus, \\
    que tirais o pecado do mundo, \\
    tende piedade de nós. \\
    Cordeiro de Deus, \\
    que tirais o pecado do mundo, \\
    dai-nos a paz.
    \vspace{.2cm} \\
    Felizes os convidados para a Ceia do Senhor.
    \vspace{.2cm} \\
    Eis o Cordeiro de Deus, \\
    que tira o pecado do mundo.
    \vspace{.2cm} \\
    {\color{VioletRed2} \Rbar.} Senhor, eu não sou digno{\color{VioletRed2}(}a{\color{VioletRed2})} \\
    de que entreis em minha morada, \\
    mas dizei uma palavra e serei salvo{\color{VioletRed2}(}a{\color{VioletRed2})}.
    \newpage
    \textcolor{VioletRed2}{Canto de comunhão}
    \vspace{.2cm} \\
    \textbf{Todo o que vive e crê em mim não morrerá eternamente!}
    \vspace{.2cm} \\
    1. Sois vós, ó Senhor, o meu Deus! \\
    Desde a aurora ansioso vos busco! \\
    A minh'alma tem sede de vós, \\
    como terra sedenta e sem água! \\
    2. Vosso amor vale mais do que a vida: \\
    e por isso meus lábios vos louvam. \\
    Quero, pois, vos louvar pela vida \\
    e elevar para vós minhas mãos! \\
    3. A minh'alma será saciada, \\
    como em grande banquete de festa; \\
    cantará a alegria em meus lábios, \\
    ao cantar para vós meu louvor! \\
    4. Para mim fostes sempre um socorro; \\
    de vossas asas à sombra eu exulto! \\
    Minha alma se agarra em vós; \\
    com poder vossa mão me sustenta.
    \vspace{.2cm} \\
    \textcolor{VioletRed2}{Canto pós-comunhão}
    \vspace{.2cm} \\
    1. A sombra vai se abrindo, quando a noite cai \\
    E vão fugindo tantas luzes \\
    De um dia, que jamais há de se acabar; \\
    De um dia, que há de começar sempre; \\
    Porque sabemos que uma nova vida, \\
    Aqui nascida, ninguém mais cancelará.
    \vspace{.2cm} \\
    \textbf{Se tu vais agora, anoitecerá \\
        Se tu vais embora, Senhor, o que será? \\
        Se tu vais agora, anoitecerá; \\
        Mas se permaneces, a noite não virá.}
    \vspace{.2cm} \\
    2. Como o mar se espraia, infinitamente, \\
    O vento soprará e abrirá os caminhos scondidos. \\
    Tantos corações hão de ver uma nova luz clara, \\
    Como uma chama que, onde passa, queima. \\
    O Teu amor todo mundo invadirá. \\
    3. A humanidade luta, sofre e espera. \\
    È terra seca e no céu não há nuvens, \\
    Mas a vida não lhe faltará; \\
    E a esperança brilhará para sempre. \\
    Contigo unidos, oh! Fonte de água viva, \\
    Tua presença o deserto acabará.
    \newpage
    \textcolor{VioletRed2}{Depois da comunhão}
    \vspace{.2cm} \\
    Oremos.
    \vspace{.2cm} \\
    Saciados pelo dom que nos salva, \\
    imploramos, Senhor, a vossa misericórdia, \\
    a fim de que, pelo mesmo sacramento \\
    que nos dais como alimento neste mundo, \\
    nos leveis a participar da vida eterna. \\
    Por Cristo, nosso Senhor.
    \vspace{.2cm} \\
    {\color{VioletRed2} \Rbar.} Amém.
\end{flushleft}
\begin{center}
    \textbf{Ritos Finais}
\end{center}
\begin{flushleft}
    \textcolor{VioletRed2}{Benção}
    \vspace{.2cm} \\
    O Senhor esteja convosco.
    \vspace{.2cm} \\
    {\color{VioletRed2} \Rbar.} Ele está no meio de nós.
    \vspace{.2cm} \\
    Sede propício, Senhor, ao vosso povo, \\
    para que, rejeitando sempre o que vos desagrada, \\
    se alegre em cumprir os vossos mandamentos. \\
    Por Cristo, nosso Senhor. \\
    {\color{VioletRed2} \Rbar.} Amém.
    \vspace{.2cm} \\
    E a bênção de Deus todo-poderoso, \\
    Pai e Filho \grecrossRed{} e Espírito Santo, \\
    desça sobre vós e permaneça para sempre.
    \vspace{.2cm} \\
    {\color{VioletRed2} \Rbar.} Amém.
    \vspace{.2cm} \\
    Ide em paz, \\
    e o Senhor vos acompanhe.
    \vspace{.2cm} \\
    {\color{VioletRed2} \Rbar.} Graças a Deus.
    \vspace{.2cm} \\
    \textcolor{VioletRed2}{Canto Final}
    \vspace{.2cm} \\
    1. Conduzidos a este deserto, (cf Mc 1, 13) \\
    Deus nos chama à libertação (cf Ex 3,8; 20,2) \\
    da indiferença e divisão: \\
    “Onde está tua irmã, teu irmão?” (cf Gn 4,9) \\
    “Eis a hora! O Reino está perto, \\
    Crê na Palavra e na conversão. (Mc 1,15)
    \vspace{.2cm} \\
    \textbf{“Vós sois todos irmãos e irmãs” (Mt 23,8) \\
        é palavra de Cristo, o Senhor; \\
        pois a fraternidade humana \\
        deve ser conversão e valor. \\
        Seja este um tempo propício (cf 2Cor 6,2) \\
        para abri-nos, enfim, ao amor!}
    \vspace{.2cm} \\
    2. A Quaresma nos chama a assumir \\
    um amor que supera barreiras, (FT 1) \\
    desejando abraçar e acolher, (FT 3) \\
    se estendendo além das fronteiras, (FT 99) \\
    rompendo as cadeias que isolam, \\
    construindo relações verdadeiras. (FT 62) \\
    3. Misericórdia, pecamos, Senhor, (Sl 50,3) \\
    sem no outro um irmão enxergar. \\
    Mas queremos vencer os conflitos, \\
    pela cultura do encontro lutar. (FT 30) \\
    Em unidade na pluralidade, \\
    um só Corpo queremos formar! (cf 1Cor 12,12-31) \\
    4. O Senhor nos propõe aliança (Gn 9,8-15) \\
    e nos trata com terno carinho. (Sl 102,4) \\
    Superemos divisões, extremismos \\
    ninguém vive o chamado sozinho. (FT 32) \\
    Só assim plantaremos a paz: \\
    “Corações ardentes e pés a caminho” (cf Lc 24, 32-33) \\
    5. “Alarga o espaço da tenda” (Is 54,2) \\
    e promove a amizade social, (cf EG 228) \\
    vence as sombras dum mundo fechado, \\
    construindo Igreja sinodal. \\
    Convertidos, renovados veremos \\
    novo céu, nova terra, afinal. (Ap 21,1-7)
\end{flushleft}
\end{document}

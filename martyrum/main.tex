\documentclass{book}
\usepackage{import}
\import{/home/vatican/livreto-liturgico/martyrum/modules/}{def-packages}
\import{/home/vatican/livreto-liturgico/martyrum/modules/}{def-macros}
\geometry{a4paper,hdivide={1.5cm,*,1.5cm},vdivide={1.5cm,*,1.5cm}}
\begin{document}
\pagestyle{empty}
\begin{center}
    \textcolor{red}{Primeira Leitura}
\end{center}
\begin{flushright}
    \textit{Zacarias, \\ a quem mataste entre o santuário e o altar \\ \textcolor{red}{(Mt 23,35).}}
\end{flushright}
\begin{flushleft}
    \vspace{.2cm}
    Leitura do Segundo Livro das Crônicas
    \hspace{\fill}
    \textcolor{red}{2Cr 24,18-22}
    \vspace{.2cm} \\
    Toda a terra tinha um só linguagem \\
    e servia-se das mesmas palavras. \\
    E aconteceu que, partindo do oriente, \\
    os homens acharam uma planície na terra de Senaar, \\
    e aí se estabeleceram. \\
    E disseram uns aos outros: \\
    ``Vamos, façamos tijolos e cozamos-los ao fogo''. \\
    Usaram tijolos em vez de pedra, \\
    e betume em lugar de argamassa. \\
    e Disseram: \\
    ``Vamos, façamos para nós uma cidade \\
    e uma torre cujo cimo atinja o céu. \\
    Assim, ficaremos famosos, \\
    e não seremos dispersos por toda a face da terra''. \\
    Então o Senhor desceu para ver a cidade \\
    e a torre que os homens estavam construindo. \\
    E o Senhor disse: \\
    ``Eis que eles são um só povo e falam uma só língua, \\
    Agora, nada os impedirá de fazer o que se propuseram. \\
    Desçamos e confundamos a sua língua \\
    de modo que não se entendam uns aos outros''. \\
    E o Senhor os dispersou daquele lugar \\
    por toda a superfície da terra, \\
    e eles cessaram de construir a cidade. \\
    Por isso, foi chamada Babel, porque foi aí que o Senhor \\
    confundiu a linguagem de todo o mundo, \\
    e daí dispersou os homens por toda terra.
    \vspace{.1cm} \\
    Palavra do Senhor \\
    {\color{red} \Rbar.} Graças a Deus.
    \vspace{.2cm} \\
    \textcolor{red}{Salmo Responsorial
        \hspace{\fill} Sl 32(33),10--11.12--13.14--15}
    \vspace{.1cm} \\
    {\color{red} \Rbar.} Feliz o povo que o Senhor escolheu por sua herança!
    \hspace{\fill}
    {\color{red} (\Rbar. 12b)}
    \vspace{.1cm} \\
    1. O Senhor desfaz os planos das nações \\
    e os projetos que os povos se propõem. \\
    Mas os desígnios do Senhor são para sempre, \\
    e os pensamentos que ele traz no coração, \\
    de geração em geração, vão perdurar.
    \hspace{\fill}{\color{red} \Rbar.}
    \vspace{.1cm} \\
    2. Feliz o povo cujo Deus é o Senhor, \\
    e a nação que escolheu por sua herança! \\
    Dos altos céus o Senhor olha e observa; \\
    ele se inclina para olhar todos os homens.
    \hspace{\fill}{\color{red} \Rbar.}
    \vspace{.1cm} \\
    3. Ele contempla do lugar onde reside \\
    e vê a todos os que habitam sobre a terra. \\
    Ele formou o coração de cada um \\
    e por todos os seus atos se interessa.
    \hspace{\fill}{\color{red} \Rbar.}
    \newpage
    \textcolor{red}{Oração}
    \vspace{.1cm} \\
    Oremos.
    \vspace{.1cm}\\
    Ó Deus todo-poderoso, \\
    que a vossa Igreja seja sempre aquele povo santo, \\
    reunido na unidade do Pai, do Filho e do Espírito Santo, \\
    que manifesta ao mundo \\
    o mistério da vossa santidade e unidade, \\
    levando as pessoas à perfeição do vosso amor. \\
    Por Cristo, nosso Senhor. \\
    {\color{red} \Rbar.} Amém.
    \vspace{.2cm} \\
\end{flushleft}
\end{document}
